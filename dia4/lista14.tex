%% This document created by Scientific Word (R) Version 3.0

\documentclass[thmsa,12pt]{article}
\usepackage[latin1]{inputenc}
\usepackage{amsmath}
\usepackage{amsfonts}
\usepackage{amssymb}
\usepackage{graphicx}

\hyphenation{a-na-li-sar}
\begin{document}
\addtolength{\headsep}{-1cm}
\addtolength{\footskip}{4cm}
\title{Lista de exerc�cios}
\date{}
\maketitle


$I.$ Uma amostra de 25 observa��es de uma Normal $(\mu,16)$ foi
coletada e forneceu uma m�dia amostral de 8. Construa intervalos
com confian�a $80\%$, $85\%$, $90\%$ e $95\%$ para a m�dia
populacional. Comente as diferen�as encontradas.

\vspace{1cm}

$II.$ Ser� coletada uma amostra de uma popula��o Normal com desvio
padr�o igual a 9. Para uma confian�a de $\alpha=90\%$, determine a
amplitude do intervalo de confian�a para a m�dia populacional nos
casos em que o tamanho da amostra � 30, 50 ou 100.

\vspace{1cm}

$III.$ O consumo de combust�vel � uma vari�vel aleat�ria com
par�metros dependendo do tipo de ve�culo. Suponha que, para um
certo autom�vel, o desvio padr�o de consumo seja conhecido e igual
a 2 $km/l$. Por�m precisamos informa��es sobre o consumo m�dio.
Para tal, coletamos uma amostra de 40 autom�veis desse modelo e
observamos o seu consumo.
\begin{itemize}
    \item Quem seria um estimador do consumo m�dio para todos os
    autom�veis desse tipo?
    \item Se a amostra forneceu um consumo m�dio de 9,3 $km/l$,
    construa um intervalo de confian�a $(94\%)$ para a m�dia de
    consumo desses carros.
    \item Se a amplitude de um intervalo de confian�a, construido
    a partir dessa amostra � de 1,5; qual teria sido o coeficiente
    de confian�a?
\end{itemize}

\newpage

$IV.$ Uma amostra de trinta dias do n�mero de ocorr�ncias
policiais em certo bairro de S�o Paulo, apresentou os seguintes
resultados: 7, 11, 8, 9, 10, 14, 6, 8, 8, 7, 8, 10, 10, 14, 12,
14, 12, 9, 11, 13, 13, 8, 6, 8, 13, 10, 14, 5, 14, 10.
\begin{itemize}
  \item Fazendo as suposi��es devidas, construa um intervalo de
  confian�a para a propor��o de dias violentos (com pelo menos 12
  ocorr�ncias). Use os dois enfoques e a confian�a de $88\%$.
  \item Em um ano (360 dias) e com a mesma confian�a de $88\%$,
  qual seria a estimativa do n�mero de dias violentos nesse
  bairro?
\end{itemize}




\end{document}
